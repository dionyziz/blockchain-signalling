\begin{abstract}
  Bitcoin and related cryptocurrencies are systematically upgraded by soft and
  hard forks. The success of a fork is conditioned in the premise that it
  will be adopted by the supermajority of miners. To ensure that miners will
  indeed adopt by supermajority, a signalling protocol has been developed which
  allows miners to advertise their intention (or not) to adopt a protocol
  change. If the signalling ratio exceeds a particular threshold within a given
  election period, the protocol change is adopted. This is termed a
  miner-activated soft fork (MASF).

  In this paper, we study the relationship between the \emph{apparent} signaling
  ratio reported on the blockchain and the \emph{actual} opinion of miners. We
  prove that, even when all honest miners vote according to their opinion, the
  apparent ratio can be made to diverge significantly from the actual opinion
  with overwhelming probability by attacks which we term \emph{election
  meddling} attacks. These attacks, which are similar but different from
  \emph{chain quality} attacks allow an adversarial minority miner to skew the
  election results in favour of their preference. We establish a lower and an
  upper bound on how much election meddling can be done by an adversary. We then
  discuss two protocols for mitigating the problem: First, a secret ballot
  mechanism which requires miners to hide their vote, and, secondly, a protocol
  which mines for votes and for blocks independently. We show, unlike deployed
  systems, the latter protocol allows the apparent opinion to really be used as
  a vote tally indicating apparent preference of the underlying computational
  power. Finally, we provide a straight-forward construction to deploy the
  mechanism in the existing Bitcoin blockchain without any forks.
\end{abstract}
